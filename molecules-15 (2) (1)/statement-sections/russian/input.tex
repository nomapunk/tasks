В первой строке теста поступает число $t$ - количество проведенных экспериментов.

Затем идет описание экспериментов, каждый эксперимент описывается в две строки:

 
В первой строке описания эксперимента поступает число $n$ - количество различных уровней молекул. 

Во второй же поступает массив $a$ длины $n$, где $a[i]$ - количество в утраченной схеме молекул уровня $i$.

В первом тесте t = 3, за каждую правильную схему к эксперименту начисляется 10 баллов. Ограничения к первому тесту:

$1 \le n \le 100$

$1 \le a[i] \le 100$


Во первом тесте t = 7, за каждую правильную схему к эксперименту начисляется 10 баллов.
Ограничения ко второму тесту:

$1 \le n \le 10^6$

$1 \le a[i] \le 10^6$

Если вы хотите набрать частичные баллы по какому-то тесту, выведете вместо схемы -1.

